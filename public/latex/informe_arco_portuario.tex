\documentclass[11pt,spanish,a4paper] {article}
\def\spanishoptions{mexico}
\usepackage[T1]{fontenc}
\usepackage{mathptmx}
\usepackage{mwe,tikz}
\usepackage[percent]{overpic}
\usepackage[margin=1.27cm,a4paper]{geometry}
%~ \usepackage[spanish,es-noshorthands,es-ucroman]{babel}
\usepackage[spanish]{babel}
\usepackage{csvsimple}
\selectlanguage{spanish}
\usepackage[utf8]{inputenc}
%~ \usepackage{fancyhdr}
%~ \usepackage[cm]{fullpage}
\usepackage{tabu}
\usepackage{hyperref}
\pagestyle{empty}
\usepackage[section]{placeins}
\usepackage{titlesec}
\titlespacing*{\subsection}{0pt}{0pt}{0pt}
\usepackage{multirow}
\usepackage{graphicx}
\usepackage{makecell}
\usepackage{floatrow}
\usepackage{caption}
\usepackage{xcolor,colortbl}
\definecolor{Black}{gray}{0.1}

\makeatletter
\g@addto@macro{\endtabular}{\rowfont{}}% Clear row font
\makeatother
\newcommand{\rowfonttype}{}% Current row font
\newcommand{\rowfont}[1]{% Set current row font
\gdef\rowfonttype{#1}#1\ignorespaces%
}
\makeatother

%~ \usepackage{titling}
%~ \setlength{\droptitle}{-3cm}
%~ \setlength{\parindent}{0mm}
\makeatletter
\def\input@path{{./public/latex/}}
%or: \def\input@path{{/path/to/folder/}{/path/to/other/folder/}}
\makeatother
\DeclareRobustCommand{\fecha}{\input{fecha}}
\DeclareRobustCommand{\fechaformat}{\input{fecha_format}}
\DeclareRobustCommand{\fechapasada}{\input{fecha_pasada}}
\DeclareRobustCommand{\situaciongeneral}{\input{situacion_general}}
\DeclareRobustCommand{\textomapasemanal}{\input{texto_mapa_semanal}}
\DeclareRobustCommand{\tendenciaclimatica}{\input{tendencia_climatica}}
\DeclareRobustCommand{\pronosticometeorologico}{\input{pronostico_meteorologico}}
\DeclareRobustCommand{\perspectivahidrometrica}{\input{perspectiva_hidrometrica}}
\DeclareRobustCommand{\pronosemanal}{\input{prono_semanal}}
\providecommand{\tightlist}{%
  \setlength{\itemsep}{0pt}\setlength{\parskip}{0pt}}
%~ \headheight 0pt
\setlength{\footskip}{10pt}
%~ \headsep 1pt 
\titlespacing*{\section}{0pt}{0ex}{0ex}
\graphicspath{{/home/leyden/44-NODEJS_APIS/informe_complementario/public/latex/}}
\usepackage{tabularx}
\begin{document}
\captionsetup{labelformat=empty}
%~ \paragraph*{}
\begingroup
\begin{center}
\renewcommand{\arraystretch}{0.8}
\begin{tabularx}{\textwidth}{|c|>{\centering\arraybackslash}X|c|}
	\hline
	\multirow{5}{*}{\includegraphics[height=1.7cm]{escudo_argentina.png}} & \textbf{\small{Ministerio de Obras Públicas}} & \multirow{5}{*}{\includegraphics[width=2.6cm,height=1.3cm]{logo_ina_crop.png}}  \\
	& \textbf{\small{Secretaría de Infraestructura y Política Hídrica}} & \\
	& \textbf{\small{Subsecretaría de Obras Hidráulicas}} & \\
	& \textbf{\small{Instituto Nacional del Agua}} & \\ 
	& \textbf{\small{2023 - 40 años de democracia}} & \\ 
	\hline
	   \makecell{\large{\textbf{Informe de}} \\ \large{\textbf{Actualización}}} &  \makecell{\large{\textbf{ARCO PORTUARIO RÍO PARANÁ}}  \\ \textbf{Pronósticos de Alturas Medias Semanales}} & Fecha: \fechaformat \\
	\hline
\end{tabularx}
\end{center}
\endgroup
%~ \paragraph*{}
%~ \vspace{1cm}
\subsection*{Situación General}
\situaciongeneral

%~ \begin{center}
%~ \begin{tabular}{|p{0.5\linewidth}|p{0.5\linewidth}|}
	%~ \hline
	%~ \includegraphics[height=6cm]{synop_pasada.png} & \textomapasemanal \\
	%~ Lluvias semanales acumuladas al \fecha & \\
	%~ \hline
%~ \end{tabular}
%~ \end{center}
\begin{figure}[H]
	\floatbox[{\capbeside\thisfloatsetup{capbesideposition={right,center},capbesidewidth=8cm}}]{figure}[\FBwidth]
	{\caption{Lluvias semanales acumuladas al \fechaformat\\\hspace{\textwidth}\\\hspace{\textwidth}\textomapasemanal}\label{synopmap}}
	{\includegraphics[height=8cm]{synop_presente.png}}
\end{figure}

%~ \subsection*{Tendencia Climática}
%~ \begin{center}
%~ \begin{tabular}{|c|c|}
	%~ \hline
	%~ \includegraphics[height=7cm]{smn_map_file.png} & \tendenciaclimatica \\
	%~ \hline
%~ \end{tabular}
%~ \end{center}
\begin{figure}[H]
	\floatbox[{\capbeside\thisfloatsetup{capbesideposition={right,center},capbesidewidth=8cm}}]{figure}[\FBwidth]
	{\caption{\large{\textbf{Tendencia Climática}}\\\hspace{\textwidth}\\\hspace{\textwidth}\tendenciaclimatica}\label{smnmap}}
	{\includegraphics[width=8cm]{smn_map_file.png}}
\end{figure}
\subsection*{Pronóstico meteorológico}
\pronosticometeorologico
\subsection*{Perspectiva hidrométrica}
\perspectivahidrometrica

\noindent
{\small
\pronosemanal
}
\end{document}

