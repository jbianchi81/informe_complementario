\documentclass[11pt,spanish] {article}
\def\spanishoptions{mexico}
\usepackage{mathptmx}
\usepackage{mwe,tikz}
\usepackage[percent]{overpic}
\usepackage[margin=0.5in]{geometry}
%~ \usepackage[spanish,es-noshorthands,es-ucroman]{babel}
\usepackage[spanish]{babel}
\usepackage{csvsimple}
\selectlanguage{spanish}
\usepackage[utf8]{inputenc}
%~ \usepackage{fancyhdr}
\usepackage[cm]{fullpage}
\usepackage{tabu}
\usepackage{hyperref}
%~ \pagestyle{fancy}
\usepackage[section]{placeins}
\usepackage{titlesec}
\usepackage{multirow}
\usepackage{graphicx}
\usepackage{makecell}
\usepackage{floatrow}
\usepackage{caption}
\usepackage{xcolor,colortbl}
\usepackage{graphbox}
%~ \usepackage[many]{tcolorbox}
%~ \usepackage{titling}
%~ \setlength{\droptitle}{-3cm}
\makeatletter
\def\input@path{{./public/latex/}}
%or: \def\input@path{{/path/to/folder/}{/path/to/other/folder/}}
\makeatother
\DeclareRobustCommand{\fecha}{\input{fecha}}
\DeclareRobustCommand{\fechaformat}{\input{fecha_format}}
\DeclareRobustCommand{\fechapasada}{\input{fecha_pasada}}
\DeclareRobustCommand{\pronodiariocomentario}{\input{prono_diario_comentario}}
\DeclareRobustCommand{\comentariofinal}{\input{comentario_final}}
\DeclareRobustCommand{\fechaproximo}{\input{fecha_proximo}}
\DeclareRobustCommand{\synoptext}{\input{synop_text}}
\DeclareRobustCommand{\textoanomalia}{\input{texto_anomalia}}
\DeclareRobustCommand{\pronodiario}{\input{prono_diario}}
\DeclareRobustCommand{\pronomensual}{\input{prono_mensual}}
\DeclareRobustCommand{\horizonte}{\input{horizonte}}
\providecommand{\tightlist}{%
  \setlength{\itemsep}{0pt}\setlength{\parskip}{0pt}}
%~ \headheight 0pt
\setlength{\footskip}{10pt}
%~ \headsep 1pt 
\titlespacing*{\section}{0pt}{0ex}{0ex}
\graphicspath{{/home/alerta5/44-NODEJS_APIS/informe_complementario/public/latex/}}
\usepackage{tabularx}
\begin{document}
\captionsetup{labelformat=empty}
%~ \paragraph*{}
%~ \begin{center}
%~ \begin{tabularx}{\textwidth}{|c|>{\centering\arraybackslash}X|c|}
	%~ \hline
	%~ \multirow{3}{*}{\includegraphics[width=5cm]{Logo_MOSP_2020.png}} & \textbf{\large{Instituto Nacional del Agua}} & \multirow{3}{*}{\includegraphics[width=2.6cm,height=1.5cm]{logo_ina_crop.png}}  \\
	%~ & \textbf{\large{Subgerencia de Sistemas de Información y}} & \\ 
	%~ & \textbf{\large{Alerta Hidrológico}} & \\
	%~ \hline
	   %~ \makecell{\Large{\textbf{Informe de Situación}}} &  \makecell{\Large{\textbf{Cuenca del río Paraguay}}} & \fecha \\
	%~ \hline
%~ \end{tabularx}

%~ Ing. Juan Borús, Lic. Maximiliano Vita Sánchez,Dr. Leandro Giordano, Lic. Juan Bianchi
%~ \end{center}
\begingroup
\begin{center}
\renewcommand{\arraystretch}{0.8}
\begin{tabularx}{\textwidth}{|c|>{\centering\arraybackslash}X|c|}
	\hline
	\multirow{5}{*}{\includegraphics[height=1.7cm]{escudo_argentina.png}} & \textbf{\small{Ministerio de Obras Públicas}} & \multirow{5}{*}{\includegraphics[width=2.6cm,height=1.3cm]{logo_ina_crop.png}}  \\
	& \textbf{\small{Secretaría de Infraestructura y Política Hídrica}} & \\
	& \textbf{\small{Subsecretaría de Obras Hidráulicas}} & \\
	& \textbf{\small{Instituto Nacional del Agua}} & \\ 
	& \textbf{\scriptsize{2022 - Las Malvinas son argentinas}} & \\ 
	\hline
	   \makecell{\textbf{Informe de Situación}} &  \makecell{\Large{\textbf{Cuenca del río Paraguay}}} &  \fecha \\
	\hline
\end{tabularx}
Ing. Juan Borús, Dr. Leandro Giordano, Mg. Juan Bianchi
\end{center}
\endgroup
%~ \paragraph*{}
%~ \vspace{1cm}
\subsection{Acumulados semanales de lluvia}
\begin{tabular}{|p{0.45\linewidth}|p{0.45\linewidth}|}
	\hline
	\includegraphics[width=8.5cm]{synop_pasada.png} & \includegraphics[width=8.5cm]{synop_presente.png} \\
	\hline
	Lluvias semanales acumuladas al \fechapasada & Lluvias acumuladas semanales al \fechaformat \\
	\hline
	\includegraphics[align=t,width=8.5cm]{mapa_anomalia.png} & \textoanomalia \\
	\hline
\end{tabular}

\begin{minipage}{\textwidth}
Los niveles esperados son (en metros):

\begin{center}

\pronodiario

\emph{En sombreado niveles por encima de los respectivos de alerta (amarillo) o evacuación (rojo)}
\end{center}
\end{minipage}

\pronodiariocomentario

Para el recinto de defensa de la ciudad de Formosa se actualiza la condición de las estaciones
de bombeo en función del nivel fluvial, actualizando la siguiente tabla:
\begin{center}
\includegraphics[width=13.5cm]{tabla_bombas.png}
\end{center}
Los niveles medios mensuales esperados son (en metros, horizonte \horizonte):
\begin{center}
\pronomensual
\end{center}
\comentariofinal

El monitoreo permanente continuará en los próximos días y se informará sobre los cambios en la
expectativa y pronósticos hidrométricos. La actualización es diaria. El mensaje completo será
preparado el próximo \emph{\fechaproximo} y presentado en la página web.

\pagebreak
\begin{center}
\includegraphics[width=17cm]{niveles_bneg.png}

\includegraphics[width=17cm]{niveles_conc.png}

\begin{figure}[b]
	\includegraphics[width=8cm]{../GRAFICOS_MULTIYEAR/BNEG.png}
	\includegraphics[width=8cm]{../GRAFICOS_MULTIYEAR/CONC.png}
\end{figure}

\includegraphics[width=17cm]{niveles_pilc.png}

\includegraphics[width=17cm]{niveles_form.png}

\begin{figure}[b]
	\includegraphics[width=8cm]{../GRAFICOS_MULTIYEAR/PILC.png}
	\includegraphics[width=8cm]{../GRAFICOS_MULTIYEAR/FORM.png}
\end{figure}

\end{center}
\end{document}

